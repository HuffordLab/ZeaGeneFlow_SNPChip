\documentclass[12pt]{amsart}
\usepackage[margin=1in]{geometry}                % See geometry.pdf to learn the layout options. There are lots.
\geometry{letterpaper}                   % ... or a4paper or a5paper or ... 
%\geometry{landscape}                % Activate for for rotated page geometry
%\usepackage[parfill]{parskip}    % Activate to begin paragraphs with an empty line rather than an indent
\usepackage{graphicx}
\usepackage{color}
\usepackage{amssymb}
\usepackage{epstopdf}
\newcommand{\mbh}[1]{\textcolor{red}{ \emph{\scriptsize  #1}} }

\DeclareGraphicsRule{.tif}{png}{.png}{`convert #1 `dirname #1`/`basename #1 .tif`.png}

\title{Personal, Relevant Background \& Future Goals Statement: David E. Hufnagel}
%\author{David E. Hufnagel}
%\date{}                                           % Activate to display a given date or no date

\begin{document}
\maketitle
It is August 2008, and there is a little blue house on an empty street speckled with houses and surrounded by forest. 
Just down the road farmers break the soil and their backs every year to turn sunlight into food for their families and the community, but this little blue house does not contain farmers, but rather a UPS driver, a stay-at-home-mom and their three boys, of which I was the middle child, then preparing to leave the countryside for Michigan State University (MSU), one of the largest universities in the United States.

Nobody in my family expected me to pursue higher education when I was growing up.  
%Neither of my parents has completed a four-year degree and my parents were merely hoping I could graduate high school and stay out of correctional institutions.  
Early in adolescence I was socially awkward, angry at the world, and convinced I was stupid.  
In school I was attacked nearly every week, often by groups.  
%Kids threw bricks at my head, jumped me in groups and slammed my face into lockers.  
I responded by fighting back and became a short-fused explosive who would start brawls over any snide remark.
At the same time I was bored with my classes and rarely tried.  
Everyone around me soon accepted my ineptitude, but after years of inadequate grades and low self-esteem I decided I would put all my efforts into school to see if I could prove everyone wrong.  
Before long I was getting along with my peers and insisting on being placed on the advanced track.  
I had found something I was good at, and it fueled me.  
Furthermore, in tenth grade I met a biology teacher who exposed me to concepts in biology for the first time since I started paying attention.  
I was so captivated that I regularly stayed after class with additional questions and stayed after school for his Microbiology Club.  

Because it is a large university with ongoing research in numerous labs, I decided to pursue my newfound thirst for biological inquiry at MSU.
In my first year I had the opportunity to get involved in a genetics lab under the guidance of Dr. Cornelius Barry. 
In his lab, I sought to determine what genetic factors cause certain fruit ripening and terpene production phenotypes in both  \textit{Solanum lycopersicum} (tomato) and its wild relatives.  
Our research on terpene production was ultimately published (Gonzales-Vigil et al. 2012).  
%The paper attributed some of the intra-specific variation in terpene synthesis in the glandular trichomes of \textit{S. habrochaites} to sequence variation in the TPS20 locus.  
While I enjoyed doing my work with Dr. Barry, I was also interested in exploring computational biology, so I took three classes in python and C++, and one introductory bioinformatics class.

In addition to taking classes, I wanted to do computational research to further explore my interest.  
I earned a position in Dr. Shin-han Shiu's lab where I acquired more independent,  computational experience including playing a significant role in a project studying how whole genome duplication has affected gene evolution in the family Brassicaceae.  
As part of this project, we assembled the \textit{Raphanus raphanistrum} (wild radish) genome and transcriptome from next-gen sequencing data.  
This data was used, along with genomic data from \textit{Arabidopsis thaliana}, \textit{A. lyrata} and \textit{Brassica rapa}, to answer questions regarding the evolution of genome structure, duplicate genes and pseudogenes in plants and was eventually published (Moghe et al. 2014). 
%Moreover, I had the opportunity to play an active role in the writing process.  
%I was exclusively responsible for several parts of the methods and results sections. 
%This research helps us understand plant evolution better, and specifically what happens to genes after they are duplicated as the evolutionary forces acting on them are changed instantly.  
I was also the leader of a project involving the acquisition and comparison of genomic pseudogenes from 31 representative species across land plants in order to learn more about the nature of pseudogenes and their role in plant evolution.  
%For this project, I planned to compare pseudogene length, distribution, and rate of evolution across land plants to see how these factors relate to species age, gene family size, and timing since whole genome duplication.  
%While significant progress was made, the project was unfinished when I left the Shiu lab to pursue my PhD.  
In addition to these projects, I was able to use my experience with pseudogene analyses and other computational skills to help with three completed projects, two of which are currently under review (Campbell et al. 2014, Wu et al., Lehti-Shiu et al.).

In addition to research and classes, at MSU I graduated in the Honors College with a 3.74 GPA while participating in other career building experiences.  
I tutored students in three science classes both with and without pay.
I also gave three oral and four poster presentations on my research, excluding lab meetings.
Furthermore, I had the opportunity to be a part of a study abroad program where I traveled with a group of students to Nicaragua for nine days to learn about both tropical ecology and Nicaraguan culture. 
This experience allowed me to see the world from a different perspective and showed me that my own was quite limited.
Moreover, I had practice in competing for funding:
During my time at MSU I was regularly on the Dean's list and earned seven competitive awards and scholarships including the 2010 Plant Genomics at MSU Summer Internship. 
My experience at MSU and my acquired taste for research inspired me to pursue graduate school, becoming the first in my family to do so.

Iowa State University (ISU) seemed like a great place to seek my PhD because of its large and longstanding Bioinformatics community, many of which had fascinating research opportunities I wanted to take advantage of.  
This growing field still has few programs, yet ISU ties Boston University with the oldest Bioinformatics graduate program in the United States.
Part of what made graduate school possible for me was earning the Plant Sciences Institute Fellowship and the Brown Fellowship which helped defray the costs of education and living at ISU.  
In my first year at ISU I earned a travel award for a presentation on my lab rotation research.
I also volunteered for the role of Director of Outreach for the Bioinformatics and Computational Biology program's Graduate Student Organization (BCBGSO).
As the Director of Outreach I felt a responsibility to assist the many students on campus who had difficulties utilizing the computational tools needed to carry out their research and coursework.
%This need first became clear when students from ISU's genetics program told us that they were struggling in classes that required working with LINUX and that they would benefit from LINUX training.
I therefore led a team of student teachers and volunteers to design, advertise, and implement two four-hour LINUX workshops in August, 2014.
The workshops were not only a great success in terms of learning outcomes for students but also in terms of my development as a leader who can organize people with diverse skill sets and points of view to accomplish a greater task.
I intend to repeat these workshops, in addition to a Python workshop, with a larger audience in January, 2015.
In addition to developing workshops, I had the opportunity to be a teaching assistant in an introductory biology lab course.
This experience required me to be a leader in the classroom on a regular basis and was a catalyst in my development of emotional maturity in dealing with grading, cheaters, and difficult people in a professional way.

Despite many great options, after research rotations I decided to work with Dr. Matthew Hufford for my PhD.
During rotations, I realized that after working in plant labs largely by coincidence I had fallen in love with plants.
In my first year I have already contributed to a finished project regarding the origin and evolution of maize in the American Southwest (Da Fonseca et al.2014).
I acquired a fascination with teosinte (i.e. wild maize) because of the opportunities to study admixture along with being able to utilize the vast resources available in maize. 
I'm interested in admixture because it seems to be a substantial yet underappreciated force in plant evolution (see Graduate Research Plan Statement).
During my PhD I intend to continue to develop my independence and leadership skills through project development and teaching.  
I am also committed to developing research skills through experience and mentoring.  
At this time I consider it likely that I will work in industry after my education, so in order to learn about the culture of industry research I plan to do an internship with one of the many seed companies that have major research branches in Iowa.
This will also hopefully serve to facilitate future collaborations between the company I intern with and ISU.

After my education I plan to do a post-doctorate abroad, after which I plan to work for a large seed company.
I hope that, if done in a compelling way, my research on the affects of admixture on the evolution of teosinte, which is closely related to the agricultural powerhouse maize, will attract seed companies to me.
Doing a post-doctorate abroad should provide me new perspectives on research and lifestyle.
%I also expect it would introduce me to collaborative opportunities that would otherwise be unavailable to me.
In larger companies like DuPont Pioneer and Syngenta there are research facilities all across the world under the tutelage of the company which would enable easy and efficient international collaboration as well.
After my post-doctorate, beyond doing my research, I will always seek opportunities to share my research with other scientists and educate the community at large about science so that they may make more educated decisions at the polls and the doctor's office.
My ultimate career goal is to significantly increase the carrying capacity of the world to allow humans more time to develop a sustainable long-term relationship with planet Earth.

In conclusion, I believe I should receive this grant because I have the right experience, skills, and personality to use this money to launch a career in science that will contribute significantly to international research, education, and innovation.  
My education and research experience have and will continue to provide me the knowledge needed to be a great scientist.  
Because of the big data revolution, my programming classes and computational research experience have provided me with skills that are in high demand.
Leadership experience in tutoring, teaching, and developing workshops will allow me to lead team members with disparate backgrounds and points of view toward goals in research and education.
My presentation experience has and will continue to develop my communication skills to share my research with other scientists and the greater community.
I also have experience in competing for funding through award and scholarship competition which will help me provide funding for my future research.  
Additionally, I have started to acquire a global perspective through my study abroad experience and will continue to do so if I earn a post-doctorate abroad.  
Funding me would also support my intention of bringing collaboration between industry and academia through my internship experiences.
Finally, I believe I have the right personality to become a globally engaged scientist with significant contributions to science and society: I am driven and focused enough to pursue challenging questions, curious and happy enough to enjoy doing so for the long-term, and confident and adventurous enough to keep pushing myself to do better science and learn more about this beautiful world.

\small{
\hfill\break\textbf{Works Cited: }\hfill

\textbf{1: Campbell}, Law, Holt, Stein, Moghe, \textbf{Hufnagel}, Lei, Achawanantakun, Jiao, Lawrence, Ware, Shiu, Childs, Sun, Jiang, and Yandell (2014). \textit{Plant Physiology}, 26(5):164(2):513-24. doi: 10.1104/pp.113.230144.
\textbf{2: Da Fonseca}, Smith, Wales, Cappellini, Pontus, Fumagalli, Samaniego, Caroe, Avila-Arcos,  \textbf{Hufnagel}, Korneliussen, Vieira, Jakobsson, Arriaza, Willerslev, Nielson, Hufford, Albrechtsen, Ross-Ibarra,  and Gilbert (2014). \textit{Nature Plants}
\textbf{3: Gonzalez-Vigil}, \textbf{Hufnagel}, Kim, Last, and Barry  (2012). \textit{The Plant Journal}, DOI: 10.1111/j.1365-313X.2012.05040.x 
\textbf{4: Lehti-Shiu}, Uygun, Panchy, Moghe, Fang, \textbf{Hufnagel}, Jasicki, Feig and Shiu. Evidence for a decaying \textit{Arabidopsis thaliana} transcription factor derived from whole genome duplication (Submitted)
\textbf{5: Moghe}, \textbf{Hufnagel}, Tang, Xiao, Dworkin, Town, Conner, and Shiu (2014). \textit{The Plant Cell}, 26(5):1925-1937.
\textbf{6: Wu}, \textbf{Hufnagel}, Denton and Shiu. Green algal retained duplicate genes tend to be stress response genes and experience frequent response gains (Submitted)
}
\end{document}  