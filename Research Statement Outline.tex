\documentclass[11pt]{amsart}
\usepackage{geometry}                % See geometry.pdf to learn the layout options. There are lots.
\geometry{letterpaper}                   % ... or a4paper or a5paper or ... 
%\geometry{landscape}                % Activate for for rotated page geometry
%\usepackage[parfill]{parskip}    % Activate to begin paragraphs with an empty line rather than an indent
\usepackage{graphicx}
\usepackage{color}
\usepackage{amssymb}
\usepackage{epstopdf}
\newcommand{\mbh}[1]{\textcolor{red}{ \emph{\scriptsize  #1}} }
\DeclareGraphicsRule{.tif}{png}{.png}{`convert #1 `dirname #1`/`basename #1 .tif`.png}

\title{NSF-GRFP Research Statement Outline}
\author{David E. Hufnagel}
%\date{}                                           % Activate to display a given date or no date

\begin{document}
\maketitle
%\section{}
%\subsection{}

\begin{description}
	\item[Project Background] \hfill
		\begin{itemize}
			\item greater \emph{Zea} phylogeny can be improved based on understanding of admixture with maize
			\item \emph{Zea luxurians} phylogeny  can be improved based on understanding of admixture with maize
			\item we are interested in admixture, explain how hybridization can lead to haplotype transfer between species and subspecies
			\item maize is a good study system because it's agriculturally important and has great variety over a wide area \mbh{both subspecies and species of \emph{Zea} demonstrate local adaptation and may have conferred these adaptations to maize during its spread across the Americas.}
			\item maize hybrid zones exist
			\item we have found interesting hybrid populations near proposed areas of maize domestication
			\item we have many questions about these populations
			\mbh{I would consider focusing on either dynamics across the entire \emph{Zea} genus or on the \emph{parviglumis}/\emph{mexicana} hybrid zones, with a preference toward the latter since this is the work we have proposed to take place here at ISU in the NSF-DEB grant.  Also, it would probably be a good idea to take some area of that proposal and extend it further to show how you plan to take ownership and expand on the goals of the project as part of your Ph.D.}
			\item explain our dataset
		\end{itemize}
	\item[Project description] \hfill
	\mbh{Here I would be sure to break your project plan down into clear aims/objectives that very cohesively form the overall project.  It would be ideal if the aims would flow into each other, but that's not essential}
		\begin{description}
			\item[All Zea] Goal: Determine the extent and nature of the affect of the post-domestication spread of maize on both wild and domesticated allopatric Zea populations.
				\begin{description}
					\item[Q1a] Did the spread of maize facilitate gene flow between wild allopatric Zea populations?
					\item[Q1b]  Was the successful spread of maize bolstered by introgression from wild locally adapted zea populations?
					\item[Methods: Zea Phylogeny] Treemix/Spacemix (grater phylogeny) (looking at landraces) (resolving luxerians)
					\item[Methods: Gene Flow] Treemix/Spacemix, STRUCTURE, Fst, heterozygosity
				\end{description}
			\item[Hybrid populations] Goal: Determine the taxonomy, demography and degree of local adaption of parv/mex hybrid populations in central and southern Mexico
				\begin{description}
					\item[Q2a] Where are these populations distributed across Central and Southern Mexico?
						\begin{description}
							\item[methods: find them] GBS
							\item[methods: identify them] STRUCTURE
						\end{description}
					\item[Q2b] Are these populations stable, locally adapted populations or simply a product of ongoing hybridization between neighboring Zea populations?
						\begin{description}
							\item[methods: relative fitness] common garden experiment
							\item[methods: origin] MSMC
						\end{description}
					\item[Q2c] What is the relationship of these hybrid populations with each other and their neighboring Zea populations?
						\begin{description}
							\item[methods: diversity] heterozygosity, Fst
							\item[methods: taxonomy] Dstatistic, Treemix/Spacemix
							\item[methods: gene flow] STRUCTURE, Treemix/Spacemix
						\end{description}
				\end{description}
			\item[Resources needed]
				\begin{description}
					\item[People] me, Matt Hufford, Jeff Ross-Ibara, Mexico collaborators
					\item[money] my salary, common garden materials (low cost is a selling point)
				\end{description}
		\end{description}
	\item[Project impacts] \hfill
	\mbh{So we'll need a "Broader Impacts" section that is less about the impact of the project on \textbf{science} and more about the impact of the project on \textbf{society}}
		\begin{description}
			\item[Maize Community] better greater phylogeny, better landrace phylogeny, resolved luxurians
			\item[Evolutionary Biologists] more info about the nature of hybridization and it's affect on evolution
			\item[me] I'll gain skills necessary to go on and tackle other problems
			\item[gk12]
			\item[collaboration w Mexican academics]
		\end{description}
	\item[Works Cited] 
\end{description}

\end{document}  